
\documentclass{package/notes}
\usepackage[english]{babel}
\usepackage{amssymb,amsmath,amsfonts}  %%% for maths
%%%%%%%%%%%%%%%%%%%%%%%%%%%%%%%%%%%%%
\usepackage{package/color-env}
\usepackage{lipsum}
\renewcommand\qedsymbol{$\blacksquare$}
%%%%%%%%%%%%%%%%%%%%%%%%%%%%%%%%%%%%%

\begin{document}

	\begin{titlepage} % Suppresses headers and footers on the title page
		
		\centering % Centre everything on the title page
		
		\scshape % Use small caps for all text on the title page
		
		\vspace*{\baselineskip} % White space at the top of the page
		
		%------------------------------------------------
		%	Title
		%------------------------------------------------
		
		\rule{\textwidth}{1.6pt}\vspace*{-\baselineskip}\vspace*{2pt} % Thick horizontal rule
		\rule{\textwidth}{0.4pt} % Thin horizontal rule
		
		\vspace{0.75\baselineskip} % Whitespace above the title
		
		{\huge TITLE\\} % Title
		
		\vspace{0.75\baselineskip} % Whitespace below the title
		
		\rule{\textwidth}{0.4pt}\vspace*{-\baselineskip}\vspace{3.2pt} % Thin horizontal rule
		\rule{\textwidth}{1.6pt} % Thick horizontal rule
		
		\vspace{2\baselineskip} % Whitespace after the title block
		
		%------------------------------------------------
		%	Subtitle
		%------------------------------------------------
		
		\LARGE{COURSE} 
		
		\vspace*{3\baselineskip} % Whitespace under the subtitle
		
		
		
		\vspace{0.5\baselineskip} 
		
		{\scshape   \LARGE Professor\\ } % Editor list
		
		\vspace{0.5\baselineskip} 
		
		\textit{\Large affiliation} % affiliation
		
		\vfill 
		
		%------------------------------------------------
		% Author
		%------------------------------------------------
		
		
		\vspace{0.3\baselineskip} 
		
		
		{\large Edited by\\  Your name} 
		
	\end{titlepage}
	\tableofcontents
%\newpage
\chapter{Title 1}

\section{Title 2}
\subsection{Title 3}

\begin{definition}[DEF NAME]{def:label} %%  [can be kept empty]
	\lipsum[1][1-3]
\end{definition}

\begin{theorem}[THM NAME]{thm:label}%%  [can be kept empty]
	\lipsum[2][1-3]
\end{theorem}

\begin{lemma}[LEM NAME]{lem:label}%%  [can be kept empty]
	\lipsum[3][1-3]
\end{lemma}

\begin{proposition}[PROP NAME]{prop:label}%%  [can be kept empty]
	\lipsum[4][1-3]
\end{proposition}

\begin{corollary}[COR NAME]{cor:label}%%  [can be kept empty]
	\lipsum[5][1-3]
\end{corollary}

\begin{problem}%%  [can be kept empty]
	\lipsum[1][1-3]
\end{problem}
\begin{proof}[Proof Head]
	\lipsum[1][1-10]
\end{proof}



\end{document}
.